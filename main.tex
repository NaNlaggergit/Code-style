\documentclass{article}
\usepackage[T2A]{fontenc}
\usepackage[utf8]{inputenc}
\usepackage[russian]{babel}
\usepackage{listings}

\title{Code Style}
\author{Kirill Lyashenko }
\date{February 2022}
\begin{document}

\maketitle

\section{Introduction}
\textbf{Github}: NaNlaggergit
\\\textbf{Mail}: hlo.lyashenko@yandex.ru
\section{C++}
\textbf{Header Files}
\\
\\У каждого .cpp файла должен быть .h, кроме файла с main
\\
\\\textbf{Classes}
\\
\\Названия классов начинаются с Заглавной буквы. Названия функции класса должны быть корректными(должны делать то чем называются) 
\\Пример:
\begin{lstlisting}
class Students {
public:
    string name;
    string last_name;
    int scores[5]; 
    float average_ball;
void calculate_average_ball()
        {
            int sum = 0;
            for (int i = 0; i < 5; ++i) {
                sum += scores[i];
            }
            average_ball = sum / 5.0;
        }
};
\end{lstlisting}
\textbf{libraries}
\\
\\Библиотеки подключаются в начеле файла
\\
\\\textbf{Comments}
\\
\\Каждое неочевидное объявление класса или структуры или функции должно сопровождаться комментарием, описывающим, для чего оно предназначено и как его следует использовать.
\\
\\\textbf{TODO Comments}
\\
\\Используйте TODO комментарии для кода, который нужно доработать 
\\
\\\textbf{Loops and Conditionals}
\\Циклы и условия всегда должны иметь{}.
Пример ошибки:
\begin{lstlisting}
for (int i = 0; i < 5; ++i) 
    if(i<5)
        i++;
\end{lstlisting}
После написания цикла или условия ставим фигурные скобки без отступа(для экономии времени и места)
\begin{lstlisting}
for(int i; i<10; i++){
}
\end{lstlisting}
\section{Список литературы}
1. https://google.github.io/styleguide/cppguide.html



\end{document}
